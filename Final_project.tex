% Options for packages loaded elsewhere
\PassOptionsToPackage{unicode}{hyperref}
\PassOptionsToPackage{hyphens}{url}
%
\documentclass[
]{article}
\usepackage{amsmath,amssymb}
\usepackage{lmodern}
\usepackage{iftex}
\ifPDFTeX
  \usepackage[T1]{fontenc}
  \usepackage[utf8]{inputenc}
  \usepackage{textcomp} % provide euro and other symbols
\else % if luatex or xetex
  \usepackage{unicode-math}
  \defaultfontfeatures{Scale=MatchLowercase}
  \defaultfontfeatures[\rmfamily]{Ligatures=TeX,Scale=1}
\fi
% Use upquote if available, for straight quotes in verbatim environments
\IfFileExists{upquote.sty}{\usepackage{upquote}}{}
\IfFileExists{microtype.sty}{% use microtype if available
  \usepackage[]{microtype}
  \UseMicrotypeSet[protrusion]{basicmath} % disable protrusion for tt fonts
}{}
\makeatletter
\@ifundefined{KOMAClassName}{% if non-KOMA class
  \IfFileExists{parskip.sty}{%
    \usepackage{parskip}
  }{% else
    \setlength{\parindent}{0pt}
    \setlength{\parskip}{6pt plus 2pt minus 1pt}}
}{% if KOMA class
  \KOMAoptions{parskip=half}}
\makeatother
\usepackage{xcolor}
\IfFileExists{xurl.sty}{\usepackage{xurl}}{} % add URL line breaks if available
\IfFileExists{bookmark.sty}{\usepackage{bookmark}}{\usepackage{hyperref}}
\hypersetup{
  pdftitle={Predicting Breast Cancer},
  pdfauthor={Francis Osei},
  hidelinks,
  pdfcreator={LaTeX via pandoc}}
\urlstyle{same} % disable monospaced font for URLs
\usepackage[margin=1in]{geometry}
\usepackage{graphicx}
\makeatletter
\def\maxwidth{\ifdim\Gin@nat@width>\linewidth\linewidth\else\Gin@nat@width\fi}
\def\maxheight{\ifdim\Gin@nat@height>\textheight\textheight\else\Gin@nat@height\fi}
\makeatother
% Scale images if necessary, so that they will not overflow the page
% margins by default, and it is still possible to overwrite the defaults
% using explicit options in \includegraphics[width, height, ...]{}
\setkeys{Gin}{width=\maxwidth,height=\maxheight,keepaspectratio}
% Set default figure placement to htbp
\makeatletter
\def\fps@figure{htbp}
\makeatother
\setlength{\emergencystretch}{3em} % prevent overfull lines
\providecommand{\tightlist}{%
  \setlength{\itemsep}{0pt}\setlength{\parskip}{0pt}}
\setcounter{secnumdepth}{-\maxdimen} % remove section numbering
\ifLuaTeX
  \usepackage{selnolig}  % disable illegal ligatures
\fi

\title{Predicting Breast Cancer}
\author{Francis Osei}
\date{17/12/2021}

\begin{document}
\maketitle

In this project we are going to explore the breast cancer data from the
UCI Machine learning repository
(\url{https://archive.ics.uci.edu/ml/datasets/Breast+Cancer+Wisconsin+(Diagnostic)}).
The figure below shows the the overall visualization process of our
data.

\includegraphics{Final_project_files/figure-latex/unnamed-chunk-2-1.pdf}

\begin{verbatim}
## Rows: 569
## Columns: 32
## $ id                      <dbl> 1, 2, 3, 4, 5, 6, 7, 8, 9, 10, 11, 12, 13, 14,~
## $ diagnosis               <chr> "M", "M", "M", "M", "M", "M", "M", "M", "M", "~
## $ radius_mean             <dbl> 17.990, 20.570, 19.690, 11.420, 20.290, 12.450~
## $ texture_mean            <dbl> 10.38, 17.77, 21.25, 20.38, 14.34, 15.70, 19.9~
## $ perimeter_mean          <dbl> 122.80, 132.90, 130.00, 77.58, 135.10, 82.57, ~
## $ area_mean               <dbl> 1001.0, 1326.0, 1203.0, 386.1, 1297.0, 477.1, ~
## $ smoothness_mean         <dbl> 0.11840, 0.08474, 0.10960, 0.14250, 0.10030, 0~
## $ compactness_mean        <dbl> 0.27760, 0.07864, 0.15990, 0.28390, 0.13280, 0~
## $ concavity_mean          <dbl> 0.30010, 0.08690, 0.19740, 0.24140, 0.19800, 0~
## $ `concave points_mean`   <dbl> 0.14710, 0.07017, 0.12790, 0.10520, 0.10430, 0~
## $ symmetry_mean           <dbl> 0.2419, 0.1812, 0.2069, 0.2597, 0.1809, 0.2087~
## $ fractal_dimension_mean  <dbl> 0.07871, 0.05667, 0.05999, 0.09744, 0.05883, 0~
## $ radius_se               <dbl> 1.0950, 0.5435, 0.7456, 0.4956, 0.7572, 0.3345~
## $ texture_se              <dbl> 0.9053, 0.7339, 0.7869, 1.1560, 0.7813, 0.8902~
## $ perimeter_se            <dbl> 8.589, 3.398, 4.585, 3.445, 5.438, 2.217, 3.18~
## $ area_se                 <dbl> 153.40, 74.08, 94.03, 27.23, 94.44, 27.19, 53.~
## $ smoothness_se           <dbl> 0.006399, 0.005225, 0.006150, 0.009110, 0.0114~
## $ compactness_se          <dbl> 0.049040, 0.013080, 0.040060, 0.074580, 0.0246~
## $ concavity_se            <dbl> 0.05373, 0.01860, 0.03832, 0.05661, 0.05688, 0~
## $ `concave points_se`     <dbl> 0.015870, 0.013400, 0.020580, 0.018670, 0.0188~
## $ symmetry_se             <dbl> 0.03003, 0.01389, 0.02250, 0.05963, 0.01756, 0~
## $ fractal_dimension_se    <dbl> 0.006193, 0.003532, 0.004571, 0.009208, 0.0051~
## $ radius_worst            <dbl> 25.38, 24.99, 23.57, 14.91, 22.54, 15.47, 22.8~
## $ texture_worst           <dbl> 17.33, 23.41, 25.53, 26.50, 16.67, 23.75, 27.6~
## $ perimeter_worst         <dbl> 184.60, 158.80, 152.50, 98.87, 152.20, 103.40,~
## $ area_worst              <dbl> 2019.0, 1956.0, 1709.0, 567.7, 1575.0, 741.6, ~
## $ smoothness_worst        <dbl> 0.1622, 0.1238, 0.1444, 0.2098, 0.1374, 0.1791~
## $ compactness_worst       <dbl> 0.6656, 0.1866, 0.4245, 0.8663, 0.2050, 0.5249~
## $ concavity_worst         <dbl> 0.71190, 0.24160, 0.45040, 0.68690, 0.40000, 0~
## $ `concave points_worst`  <dbl> 0.26540, 0.18600, 0.24300, 0.25750, 0.16250, 0~
## $ symmetry_worst          <dbl> 0.4601, 0.2750, 0.3613, 0.6638, 0.2364, 0.3985~
## $ fractal_dimension_worst <dbl> 0.11890, 0.08902, 0.08758, 0.17300, 0.07678, 0~
\end{verbatim}

Figure 2 shows all the features (Attributes) of our data. Each feature
has either mean, Standard error (se) and worst. This gives us a clear
idea of the features we are working with.

\includegraphics{Final_project_files/figure-latex/unnamed-chunk-5-1.pdf}

The correlation plot uses the pearson correlation to the direction and
strength of the relation of our features. This helps to reduce the
number of features in our data. Here we represented the correlation in
form of heatmap showing the range of values using colours. This
visualization is important especially when dooing regression analysis.

\includegraphics{Final_project_files/figure-latex/unnamed-chunk-6-1.pdf}

\section{Beeswarm plots}

The beeswarm is a techique to avoid overplotting. This randomly move
data points away from each other.\textbackslash{} We plotted a
quasirandom beeswarm to distinguished between Begign and Malignant. For
each plot we have 10 features. The are some features that are difficult
to differentiate between Begign and Malignant.
\includegraphics{Final_project_files/figure-latex/unnamed-chunk-7-1.pdf}

\includegraphics{Final_project_files/figure-latex/unnamed-chunk-8-1.pdf}

\includegraphics{Final_project_files/figure-latex/unnamed-chunk-9-1.pdf}

\section{Histogram}

Some plots shows clear differentiation between the Malignant and Bening
cell whiles others are very difficult to distinguish. We tried to use
different plot to see if we can get a clear picture of the the
difference between Bening and Malignant. In the histogram plot, the
observation in each bin is represented by the height of the bar. In this
plot, the variance for some features can be seen more clearly interns of
Diagnosis.
\includegraphics{Final_project_files/figure-latex/unnamed-chunk-10-1.pdf}

\section{Scatter plot with individual ellipse}

The scatter plot uses dots to represent two different numerical
features. The scatter plots shows the relationship between variable
(mean,standard error, and worst).\textbackslash{} For classification
purpose, we define the region that contains 95\% of the samples for each
diagnosis. The individual ellipse provides us with some interesting
difference between the diagnosis. \bullet The two ellipse are
overlapping and parallel. \bullet The two ellipse are overlapping and
perpendicular. \bullet The two ellipse are non overlapping and parallel.
\bullet The two ellipse are non overlapping and perpendicular.

\includegraphics{Final_project_files/figure-latex/unnamed-chunk-12-1.pdf}

\includegraphics{Final_project_files/figure-latex/unnamed-chunk-14-1.pdf}

\includegraphics{Final_project_files/figure-latex/unnamed-chunk-16-1.pdf}

\end{document}
